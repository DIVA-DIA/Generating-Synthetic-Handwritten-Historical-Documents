\documentclass[20pt]{report}
\usepackage[latin]{babel}
\usepackage[utf8]{inputenc}
\usepackage{blindtext}

\usepackage{multicol}
\setlength\columnsep{30pt}
%Lettrine and Calligraphy
\usepackage{lettrine} 
\usepackage{yfonts}
\usepackage{GoudyIn}
\usepackage{calligra}

\usepackage[utf8]{inputenc}
\usepackage{erewhon}
\usepackage{lipsum} % useless

\usepackage{pgffor}
\usepackage[x11names]{xcolor}

\renewcommand{\LettrineFontHook}{\color{VioletRed4}\GoudyInfamily{}}
\LettrineTextFont{\itshape}
\setcounter{DefaultLines}{6}%
\pdfinterwordspaceon
\pagestyle{empty}
\begin{document}

%\LettrineTextFont{\itshape}

\setcounter{DefaultLines}{5}
\begin{multicols}{2}
\foreach \n in {0,...,40}{
	
	\lettrine{G}  allia est omnis divisa in partes tres, quarum unam incolunt Belgae, aliam Aquitani, 		tertiam qui ipsorum lingua Celtae, nostra Galli appellantur. Hi omnes lingua, institutis, legibus inter se differunt. Gallos ab Aquitanis Garumna flumen, a Belgis Matrona et Sequana dividit. Horum omnium fortissimi sunt Belgae, propterea quod a cultu atque humanitate rovinciae longissime absunt, minimeque ad eos mercatores saepe commeant atque ea quae ad effeminandos animos pertinent important, proximique sunt Germanis, qui trans Rhenum incolunt, quibuscum continenter bellum gerunt.
	Qua de causa Helvetii quoque reliquos Gallos virtute praecedunt, quod fere cotidianis proeliis cum Germanis contendunt, cum aut suis finibus eos prohibent aut ipsi in eorum finibus bellum gerunt. Eorum una, pars, quam Gallos obtinere dictum est, initium capit a flumine Rhodano, continetur Garumna flumine, Oceano, finibus Belgarum, attingit etiam ab Sequanis et Helvetiis flumen Rhenum, vergit ad septentriones.
	Belgae ab extremis Galliae finibus oriuntur, pertinent ad inferiorem partem fluminis Rheni, spectant in septentrionem et orientem solem. Aquitania a Garumna flumine ad Pyrenaeos montes et eam partem Oceani quae est ad Hispaniam pertinet; spectat inter occasum solis et septentriones.
	
}
\end{multicols}



\begin{multicols}{2}
\setcounter{DefaultLines}{5}
\foreach \n in {0,...,40}{
	
	\lettrine{A} pud Helvetios longe nobilissimus fuit et ditissimus Orgetorix.
	Is M. Messala, (et P.) M. Pisone consulibus regni cupiditate inductus coniurationem nobilitatis fecit et civitati persuasit ut de finibus suis cum omnibus copiis exirent: perfacile esse, cum virtute omnibus praestarent, totius Galliae imperio potiri. Id hoc facilius iis persuasit, quod undique loci natura Helvetii continentur: una ex parte flumine Rheno latissimo atque altissimo, qui agrum Helvetium a Germanis dividit; altera ex parte monte Iura altissimo, qui est inter Sequanos et Helvetios; tertia lacu Lemanno et flumine Rhodano, qui provinciam nostram ab Helvetiis Dividit. 
	His rebus fiebat ut et minus late vagarentur et minus facile finitimis bellum inferre possent; qua ex parte homines bellandi cupidi magno dolore adficiebantur. Pro multitudine autem hominum et pro gloria belli atque fortitudinis angustos se fines habere arbitrabantur, qui in longitudinem milia passuum CCXL, in latitudinem CLXXX patebant.	
		
}
\end{multicols}



\renewcommand{\LettrineFontHook}{\color{VioletRed4}\GoudyInfamily{}}
\begin{multicols}{2}
\setcounter{DefaultLines}{5}
\foreach \n in {0,...,40}{
	
	\lettrine{H}is rebus adducti et auctoritate Orgetorigis permoti constituerunt ea quae ad proficiscendum pertinerent comparare, iumentorum et carrorum quam maximum numerum coemere, sementes quam maximas facere, ut in itinere copia frumenti suppeteret, cum proximis civitatibus pacem et amicitiam confirmare. Ad eas res conficiendas biennium sibi satis esse duxerunt; in tertium annum profectionem lege confirmant. Ad eas res conficiendas Orgetorix deligitur. Is sibi legationem ad civitates suscipit. In eo itinere persuadet Castico, Catamantaloedis filio, Sequano, cuius pater regnum in Sequanis multos annos obtinuerat et a senatu populi Romani amicus appellatus erat, ut regnum in civitate sua occuparet, quod pater ante habuerit; itemque Dumnorigi Haeduo, fratri Diviciaci, qui eo tempore principatum in civitate obtinebat ac maxime plebi acceptus erat, ut idem conaretur persuadet eique filiam suam in matrimonium dat. Perfacile factu esse illis probat conata perficere, propterea quod ipse suae civitatis imperium obtenturus esset: non esse dubium, quin totius Galliae plurimum Helvetii possent; se suis copiis suoque exercitu illis regna conciliaturum confirmat. Hac oratione adducti inter se fidem et ius iurandum dant et regno occupato per tres potentissimos ac firmissimos populos totius Galliae sese potiri posse sperant.	
	
}
\end{multicols}


\renewcommand{\LettrineFontHook}{\color{orange}\GoudyInfamily{}}
%\LettrineTextFont{\itshape}
\setcounter{DefaultLines}{5}%

\begin{multicols}{2}
\setcounter{DefaultLines}{5}
\foreach \n in {0,...,40}{
	
	\lettrine{E} a res est Helvetiis per indicium enuntiata. Moribus suis Orgetoricem ex vinculis causam dicere coegerunt; damnatum poenam sequi oportebat, ut igni cremaretur. Die constituta causae dictionis Orgetorix ad iudicium omnem suam familiam, ad hominum milia decem, undique coegit, et omnes clientes obaeratosque suos, quorum magnum numerum habebat, eodem conduxit; per eos, ne causam diceret, se eripuit. Cum civitas ob eam rem incitata armis ius suum exequi conaretur multitudinemque hominum ex agris magistratus cogerent, Orgetorix mortuus est; neque abest suspicio, ut Helvetii arbitrantur, quin ipse sibi mortem consciverit.	
	
}
\end{multicols}



\begin{multicols}{2}
\setcounter{DefaultLines}{5}
\foreach \n in {0,...,40}{
	
	\lettrine{P} ost eius mortem nihilo minus Helvetii id, quod constituerant, facere conantur, ut e finibus suis exeant. Ubi iam se ad eam rem paratos esse arbitrati sunt, oppida sua omnia, numero ad duodecim, vicos ad quadringentos, reliqua privata aedificia incendunt; frumentum omne, praeter quod secum portaturi erant, comburunt, ut domum reditionis spe sublata paratiores ad omnia pericula subeunda essent; trium mensum molita cibaria sibi quemque domo efferre iubent. Persuadent Rauracis et Tulingis et Latobrigis finitimis, uti eodem usi consilio oppidis suis vicisque exustis una cum iis proficiscantur, Boiosque, qui trans Rhenum incoluerant et in agrum Noricum transierant Noreiamque oppugnabant, receptos ad se socios sibi adsciscunt.

}
\end{multicols}



\begin{multicols}{2}
\setcounter{DefaultLines}{5}
\foreach \n in {0,...,40}{
	
	\lettrine{E} rant omnino itinera duo, quibus itineribus domo exire possent: unum per Sequanos, angustum et difficile, inter montem Iuram et flumen Rhodanum, vix qua singuli carri ducerentur, mons autem altissimus impendebat, ut facile perpauci prohibere possent; alterum per provinciam nostram, multo facilius atque expeditius, propterea quod inter fines Helvetiorum et Allobrogum, qui nuper pacati erant, Rhodanus fluit isque non nullis locis vado transitur. Extremum oppidum Allobrogum est proximumque Helvetiorum finibus Genava. Ex eo oppido pons ad Helvetios pertinet. Allobrogibus sese vel persuasuros, quod nondum bono animo in populum Romanum viderentur, existimabant vel vi coacturos, ut per suos fines eos ire paterentur. Omnibus rebus ad profectionem comparatis diem dicunt, qua die ad ripam Rhodani omnes conveniant. Is dies erat a. d. V. Kal. Apr. L. Pisone, A. Gabinio consulibus.


}
\end{multicols}


\renewcommand{\LettrineFontHook}{\color{red}\GoudyInfamily{}}

%\LettrineTextFont{\itshape}
\setcounter{DefaultLines}{5}

\begin{multicols}{2}
\foreach \n in {0,...,40}{

	\lettrine{C} aesari cum id nuntiatum esset eos per provinciam nostram iter facere conari, maturat ab urbe proficisci et quam maximis potest itineribus in Galliam ulteriorem contendit et ad Genavam pervenit. Provinciae toti quam maximum potest militum numerum imperat (erat omnino in Gallia ulteriore legio una), pontem, qui erat ad Genavam, iubet rescindi. Ubi de eius adventu Helvetii certiores facti sunt, legatos ad eum mittunt nobilissimos civitatis, cuius legationis Nammeius et Verucloetius principem locum obtinebant, qui dicerent sibi esse in animo sine ullo maleficio iter per provinciam facere, propterea quod aliud iter haberent nullum: rogare, ut eius voluntate id sibi facere liceat. Caesar, quod memoria tenebat L. Cassium consulem occisum exercitumque eius ab Helvetiis pulsum et sub iugum missum, concedendum non putabat; neque homines inimico animo, data facultate per provinciam itineris faciundi, temperaturos ab iniuria et maleficio existimabat. Tamen, ut spatium intercedere posset dum milites, quos imperaverat, convenirent, legatis respondit diem se ad deliberandum sumpturum: si quid vellent, ad Id. April. reverterentur.
	
}
\end{multicols}



\begin{multicols}{2}
\foreach \n in {0,...,40}{

	\lettrine{I} nterea ea legio, quam secum habebat, militibusque, qui ex provincia convenerant, a lacu Lemanno, qui in flumen Rhodanum influit, ad montem Iuram, qui fines Sequanorum ab Helvetiis dividit, milia passuum XVIIII murum in altitudinem pedum sedecim fossamque perducit. Eo opere perfecto praesidia disponit, castella communit, quo facilius, si se invito transire conentur, prohibere possit. Ubi ea dies, quam constituerat cum legatis, venit et legati ad eum reverterunt, negat se more et exemplo populi Romani posse iter ulli per provinciam dare et, si vim facere conentur, prohibiturum ostendit. Helvetii ea spe deiecti navibus iunctis ratibusque compluribus factis, alii vadis Rhodani, qua minima altitudo fluminis erat, non numquam interdiu, saepius noctu, si perrumpere possent, conati, operis munitione et militum concursu et telis repulsi, hoc conatu destiterunt.
	
	}
\end{multicols}



\begin{multicols}{2}
\foreach \n in {0,...,40}{

	\lettrine{R} elinquebatur una per Sequanos via, qua Sequanis invitis propter angustias ire non poterant. His cum sua sponte persuadere non possent, legatos ad Dumnorigem Haeduum mittunt, ut eo deprecatore a Sequanis impetrarent. Dumnorix gratia et largitione apud Sequanos plurimum poterat et Helvetiis erat amicus, quod ex ea civitate Orgetorigis filiam in matrimonium duxerat, et cupiditate regni adductus novis rebus studebat et quam plurimas civitates suo beneficio habere obstrictas volebat. Itaque rem suscipit et a Sequanis impetrat, ut per fines suos Helvetios ire patiantur, obsidesque uti inter sese dent perficit: Sequani, ne itinere Helvetios prohibeant, Helvetii, ut sine maleficio et iniuria transeant.
	
	}
\end{multicols}



\begin{multicols}{2}
\foreach \n in {0,...,40}{

	\lettrine{H} oc proelio facto, reliquas copias Helvetiorum ut consequi posset, pontem in Arari faciendum curat atque ita exercitum traducit. Helvetii repentino eius adventu commoti cum id quod ipsi diebus XX aegerrime confecerant, ut flumen transirent, illum uno die fecisse intellegerent, legatos ad eum mittunt; cuius legationis Divico princeps fuit, qui bello Cassiano dux Helvetiorum fuerat. Is ita cum Caesare egit: si pacem populus Romanus cum Helvetiis faceret, in eam partem ituros atque ibi futuros Helvetios, ubi eos Caesar constituisset atque esse voluisset; sin bello persequi perseveraret, reminisceretur et veteris incommodi populi Romani et pristinae virtutis Helvetiorum. Quod improviso unum pagum adortus esset, cum ii, qui flumen transissent, suis auxilium ferre non possent, ne ob eam rem aut suae magnopere virtuti tribueret aut ipsos despiceret. Se ita a patribus maioribusque suis didicisse, ut magis virtute contenderent quam dolo aut insidiis niterentur. Quare ne committeret, ut is locus, ubi constitissent, ex calamitate populi Romani et internecione exercitus nomen caperet aut memoriam proderet.
	
	}
\end{multicols}



\begin{multicols}{2}
\foreach \n in {0,...,40}{

	\lettrine{H} is Caesar ita respondit: eo sibi minus dubitationis dari, quod eas res, quas legati Helvetii commemorassent, memoria teneret, atque eo gravius ferre quo minus merito populi Romani accidissent; qui si alicuius iniuriae sibi conscius fuisset, non fuisse difficile cavere; sed eo deceptum, quod neque commissum a se intellegeret, quare timeret, neque sine causa timendum putaret. Quod si veteris contumeliae oblivisci vellet, num etiam recentium iniuriarum, quod eo invito iter per provinciam per vim temptassent, quod Haeduos, quod Ambarros, quod Allobrogas vexassent, memoriam deponere posse? Quod sua victoria tam insolenter gloriarentur quodque tam diu se impune iniurias tulisse admirarentur, eodem pertinere. Consuesse enim deos immortales, quo gravius homines ex commutatione rerum doleant, quos pro scelere eorum ulcisci velint, his secundiores interdum res et diuturniorem impunitatem concedere. Cum ea ita sint, tamen, si obsides ab iis sibi dentur, uti ea quae polliceantur facturos intellegat, et si Haeduis de iniuriis quas ipsis sociisque eorum intulerint, item si Allobrogibus satis faciunt, sese cum iis pacem esse facturum. Divico respondit: ita Helvetios a maioribus suis institutos esse uti obsides accipere, non dare, consuerint; eius rei populum Romanum esse testem. Hoc responso dato discessit.
	
	}
\end{multicols}



\begin{multicols}{2}
\foreach \n in {0,...,40}{

	\lettrine{P} ostero die castra ex eo loco movent. Idem facit Caesar equitatumque omnem, ad numerum quattuor milium, quem ex omni provincia et Haeduis atque eorum sociis coactum habebat, praemittit, qui videant, quas in partes hostes iter faciant. Qui cupidius novissimum agmen insecuti alieno loco cum equitatu Helvetiorum proelium committunt et pauci de nostris cadunt. Quo proelio sublati Helvetii, quod quingentis equitibus tantam multitudinem equitum propulerant, audacius subsistere non numquam et novissimo agmine proelio nostros lacessere coeperunt. Caesar suos a proelio continebat, ac satis habebat in praesentia hostem rapinis, pabulationibus populationibusque prohibere. Ita dies circiter XV iter fecerunt, uti inter novissimum hostium agmen et nostrum primum non amplius quinis aut senis milibus passuum interesset.	
	
	}
\end{multicols}



\renewcommand{\LettrineFontHook}{\color{blue}}
%\LettrineTextFont{\itshape}
\setcounter{DefaultLines}{3}%



\foreach \n in {0,...,40}{

	\lettrine{I} nterim cotidie Caesar Haeduos frumentum, quod essent publice polliciti, flagitare. Nam propter frigora [quod Gallia sub septentrionibus, ut ante dictum est, posita est,] non modo frumenta in agris matura non erant, sed ne pabuli quidem satis magna copia suppetebat; eo autem frumento, quod flumine Arari navibus subvexerat, propterea uti minus poterat, quod iter ab Arari Helvetii averterant, a quibus discedere nolebat. Diem ex die ducere Haedui: conferri, comportari, adesse dicere. Ubi se diutius duci intellexit et diem instare, quo die frumentum militibus metiri oporteret, convocatis eorum principibus, quorum magnam copiam in castris habebat, in his Diviciaco et Lisco, qui summo magistratui praeerat, quem vergobretum appellant Haedui, qui creatur annuus et vitae necisque in suos habet potestatem, graviter eos accusat, quod, cum neque emi neque ex agris sumi possit, tam necessario tempore, tam propinquis hostibus ab iis non sublevetur, praesertim cum magna ex parte eorum precibus adductus bellum susceperit; multo etiam gravius, quod sit destitutus, queritur.	
	
}




\begin{multicols}{2}
\foreach \n in {0,...,40}{

	\lettrine{T} um demum Liscus oratione Caesaris adductus, quod antea tacuerat, proponit: esse non nullos, quorum auctoritas apud plebem plurimum valeat, qui privatim plus possint quam ipsi magistratus. Hos seditiosa atque improba oratione multitudinem deterrere, ne frumentum conferant quod debeant: praestare, si iam principatum Galliae obtinere non possint, Gallorum quam Romanorum imperia perferre, neque dubitare [debeant], quin, si Helvetios superaverint Romani, una cum reliqua Gallia Haeduis libertatem sint erepturi. Ab isdem nostra consilia quaeque in castris gerantur hostibus enuntiari; hos a se coerceri non posse. Quin etiam, quod necessario rem coactus Caesari enuntiarit, intellegere sese, quanto id cum periculo fecerit, et ob eam causam, quam diu potuerit, tacuisse.
	
}
\end{multicols}



\begin{multicols}{2}
\foreach \n in {0,...,40}{

	\lettrine{C} aesar hac oratione Lisci Dumnorigem, Diviciaci fratrem, designari sentiebat, sed, quod pluribus praesentibus eas res iactari nolebat, celeriter concilium dimittit, Liscum retinet. Quaerit ex solo ea quae in conventu dixerat. Dicit liberius atque audacius. Eadem secreto ab aliis quaerit; 
reperit esse vera: ipsum esse Dumnorigem, summa audacia, magna apud plebem propter liberalitatem gratia, cupidum rerum novarum. Complures annos portoria reliquaque omnia Haeduorum vectigalia parvo pretio redempta habere, propterea quod illo licente contra liceri audeat nemo. His rebus et suam rem familiarem auxisse et facultates ad largiendum magnas comparasse

}
\end{multicols}



\begin{multicols}{2}
\foreach \n in {0,...,40}{

	\lettrine{M} agnum numerum equitatus suo sumptu semper alere et circum se habere, neque solum domi, sed etiam apud finitimas civitates largiter posse, atque huius potentiae causa matrem in Biturigibus homini illic nobilissimo ac potentissimo conlocasse; ipsum ex Helvetiis uxorem habere, sororem ex matre et propinquas suas nuptum in alias civitates conlocasse. Favere et cupere Helvetiis propter eam adfinitatem, odisse etiam suo nomine Caesarem et Romanos, quod eorum adventu potentia eius deminuta et Diviciacus frater in antiquum locum gratiae atque honoris sit restitutus. Si quid accidat Romanis, summam in spem per Helvetios regni obtinendi venire; imperio populi Romani non modo de regno, sed etiam de ea quam habeat gratia desperare. Reperiebat etiam in quaerendo Caesar, quod proelium equestre adversum paucis ante diebus esset factum, initium eius fugae factum a Dumnorige atque eius equitibus (nam equitatui, quem auxilio Caesari Haedui miserant, Dumnorix praeerat); eorum fuga reliquum esse equitatum perterritum.	

}
\end{multicols}


\begin{multicols}{2}
\foreach \n in {0,...,40}{

	\lettrine{Q} uibus rebus cognitis, cum ad has suspiciones certissimae res accederent, quod per fines Sequanorum Helvetios traduxisset, quod obsides inter eos dandos curasset, quod ea omnia non modo iniussu suo et civitatis, sed etiam inscientibus ipsis fecisset, quod a magistratu Haeduorum accusaretur, satis esse causae arbitrabatur, quare in eum aut ipse animadverteret aut civitatem animadvertere iuberet. His omnibus rebus unum repugnabat, quod Diviciaci fratris summum in populum Romanum studium, summum in se voluntatem, egregiam fidem, iustitiam, temperantiam cognoverat; nam, ne eius supplicio Diviciaci animum offenderet, verebatur. Itaque, priusquam quicquam conaretur, Diviciacum ad se vocari iubet et, cotidianis interpretibus remotis, per C. Valerium Troucillum, principem Galliae provinciae, familiarem suum, cui summam omnium rerum fidem habebat, cum eo conloquitur; simul commonefacit, quae ipso praesente in concilio [Gallorum] de Dumnorige sint dicta, et ostendit, quae separatim quisque de eo apud se dixerit. Petit atque hortatur, ut sine eius offensione animi vel ipse de eo causa cognita statuat vel civitatem statuere iubeat.	
	
}
\end{multicols}



\begin{multicols}{2}
\foreach \n in {0,...,40}{

	\lettrine{D} iviciacus multis cum lacrimis Caesarem complexus obsecrare coepit, ne quid gravius in fratrem statueret: scire se illa esse vera, nec quemquam ex eo plus quam se doloris capere, propterea quod, cum ipse gratia plurimum domi atque in reliqua Gallia, ille minimum propter adulescentiam posset, per se crevisset; quibus opibus ac nervis non solum ad minuendam gratiam, sed paene ad perniciem suam uteretur. Sese tamen et amore fraterno et existimatione vulgi commoveri. Quod si quid ei a Caesare gravius accidisset, cum ipse eum locum amicitiae apud eum teneret, neminem existimaturum non sua voluntate factum; qua ex re futurum, uti totius Galliae animi a se averterentur. Haec cum pluribus verbis flens a Caesare peteret, Caesar eius dextram prendit; consolatus rogat finem orandi faciat; tanti eius apud se gratiam esse ostendit, uti et rei publicae iniuriam et suum dolorem eius voluntati ac precibus condonet. Dumnorigem ad se vocat, fratrem adhibet; quae in eo reprehendat, ostendit; quae ipse intellegat, quae civitas queratur, proponit; monet, ut in reliquum tempus omnes suspiciones vitet; praeterita se Diviciaco fratri condonare dicit. Dumnorigi custodes ponit, ut, quae agat, quibuscum loquatur, scire possit.
	
}
\end{multicols}

\renewcommand{\LettrineFontHook}{\calligra}
\setcounter{DefaultLines}{7}%
%\LettrineTextFont{\itshape}
\setlength{\DefaultNindent}{0em}




\foreach \n in {0,...,40}{

	\lettrine[findent=7.0em]{B} ello Helvetiorum confecto totius fere Galliae legati, principes civitatum, ad Caesarem gratulatum convenerunt: intellegere sese, tametsi pro veteribus Helvetiorum iniuriis populi Romani ab his poenas bello repetisset, tamen eam rem non minus ex usu [terrae] Galliae quam populi Romani accidisse, propterea quod eo consilio florentissimis rebus domos suas Helvetii reliquissent, uti toti Galliae bellum inferrent imperioque potirentur, locumque domicilio ex magna copia deligerent, quem ex omni Gallia oportunissimum ac fructuosissimum iudicassent, reliquasque civitates stipendiarias haberent. Petierunt, uti sibi concilium totius Galliae in diem certam indicere idque Caesaris facere voluntate liceret: sese habere quasdam res, quas ex communi consensu ab eo petere vellent. Ea re permissa diem concilio constituerunt et iure iurando ne quis enuntiaret, nisi quibus communi consilio mandatum esset, inter se sanxerunt.
	
}




\foreach \n in {0,...,40}{

	\lettrine[findent=7.0em]{E}  o concilio dimisso, idem princeps civitatum, qui ante fuerant ad Caesarem, reverterunt petieruntque, uti sibi secreto in occulto de sua omniumque salute cum eo agere liceret. Ea re impetrata sese omnes flentes Caesari ad pedes proiecerunt: non minus se id contendere et laborare, ne ea, quae dixissent, enuntiarentur, quam uti ea quae vellent impetrarent, propterea quod, si enuntiatum esset, summum in cruciatum se venturos viderent. Locutus est pro his Diviciacus Haeduus: Galliae totius factiones esse duas; harum alterius principatum tenere Haeduos, alterius Arvernos. Hi cum tantopere de potentatu inter se multos annos contenderent, factum esse uti ab Arvernis Sequanisque Germani mercede arcesserentur. Horum primo circiter milia XV Rhenum transisse; postea quam agros et cultum et copias Gallorum homines feri ac barbari adamassent, traductos plures; nunc esse in Gallia ad C et XX milium numerum. Cum his Haeduos eorumque clientes semel atque iterum armis contendisse; magnam calamitatem pulsos accepisse, omnem nobilitatem, omnem senatum, omnem equitatum amisisse. Quibus proeliis calamitatibusque fractos, qui et sua virtute et populi Romani hospitio atque amicitia plurimum ante in Gallia potuissent, coactos esse Sequanis obsides dare nobilissimos civitatis et iure iurando civitatem obstringere sese neque obsides repetituros neque auxilium a populo Romano imploraturos neque recusaturos, quo minus perpetuo sub illorum dicione atque imperio essent. Unum se esse ex omni civitate Haeduorum qui adduci non potuerit, ut iuraret aut liberos suos obsides daret.	
	
}




\foreach \n in {0,...,40}{

	\lettrine[findent=7.0em]{O} b eam rem se ex civitate profugisse et Romam ad senatum venisse auxilium postulatum, quod solus neque iure iurando neque obsidibus teneretur. Sed peius victoribus Sequanis quam Haeduis victis accidisse, propterea quod Ariovistus, rex Germanorum, in eorum finibus consedisset tertiamque partem agri Sequani, qui esset optimus totius Galliae, occupavisset et nunc de altera parte tertia Sequanos decedere iuberet, propterea quod paucis mensibus ante Harudum milia hominum XXIIII ad eum venissent, quibus locus ac sedes pararentur. Futurum esse paucis annis, uti omnes ex Galliae finibus pellerentur atque omnes Germani Rhenum transirent; neque enim conferendum esse Gallicum cum Germanorum agro neque hanc consuetudinem victus cum illa comparandam. Ariovistum autem, ut semel Gallorum copias proelio vicerit, quod proelium factum sit ad Magetobrigam, superbe et crudeliter imperare, obsides nobilissimi cuiusque liberos poscere et in eos omnia exempla cruciatusque edere, si qua res non ad nutum aut ad voluntatem eius facta sit. Hominem esse barbarum, iracundum, temerarium: non posse eius imperia diutius sustineri. Nisi quid in Caesare populoque Romano sit auxilii, omnibus Gallis idem esse faciendum quod Helvetii fecerint, ut domo emigrent, aliud domicilium, alias sedes, remotas a Germanis, petant fortunamque, quaecumque accidat, experiantur. Haec si enuntiata Ariovisto sint, non dubitare, quin de omnibus obsidibus qui apud eum sint gravissimum supplicium sumat. Caesarem vel auctoritate sua atque exercitus vel recenti victoria vel nomine populi Romani deterrere posse, ne maior multitudo Germanorum Rhenum traducatur, Galliamque omnem ab Ariovisti iniuria posse defendere.
		
		
}





\foreach \n in {0,...,40}{

	\lettrine[findent=7.0em]{H} ac oratione ab Diviciaco habita omnes, qui aderant, magno fletu auxilium a Caesare petere coeperunt. Animadvertit Caesar unos ex omnibus Sequanos nihil earum rerum facere quas ceteri facerent sed tristes capite demisso terram intueri. Eius rei quae causa esset, miratus ex ipsis quaesiit. Nihil Sequani respondere, sed in eadem tristitia taciti permanere. Cum ab his saepius quaereret neque ullam omnino vocem exprimere posset, idem Diviacus Haeduus respondit: hoc esse miseriorem et graviorem fortunam Sequanorum quam reliquorum, quod soli ne in occulto quidem queri neque auxilium implorare auderent absentisque Ariovisti crudelitatem, velut si coram adesset, horrerent, propterea quod reliquis tamen fugae facultas daretur, Sequanis vero, qui intra fines suos Ariovistum recepissent, quorum oppida omnia in potestate eius essent, omnes cruciatus essent perferendi.
	
}


\newpage

\renewcommand{\LettrineFontHook}{\calligra}
\setcounter{DefaultLines}{5}%
%\LettrineTextFont{\itshape}
\setlength{\DefaultNindent}{0em}

\foreach \n in {0,...,40}{

	\lettrine[findent=7.0em]{H}  ac oratione habita, concilium dimisit. Et secundum ea multae res eum hortabantur, quare sibi eam rem cogitandam et suscipiendam putaret, in primis quod Haeduos, fratres consanguineosque saepe numero a senatu appellatos, in servitute atque dicione videbat Germanorum teneri eorumque obsides esse apud Ariovistum ac Sequanos intellegebat; quod in tanto imperio populi Romani turpissimum sibi et rei publicae esse arbitrabatur. Paulatim autem Germanos consuescere Rhenum transire et in Galliam magnam eorum multitudinem venire populo Romano periculosum videbat, neque sibi homines feros ac barbaros temperaturos existimabat, quin, cum omnem Galliam occupavissent, ut ante Cimbri Teutonique fecissent, in provinciam exirent atque inde in Italiam contenderent, praesertim cum Sequanos a provincia nostra Rhodanus divideret; quibus rebus quam maturrime occurrendum putabat. Ipse autem Ariovistus tantos sibi spiritus, tantam arrogantiam sumpserat, ut ferendus non videretur.
	
}



\foreach \n in {0,...,40}{

	\lettrine[findent=5.0em]{Q} uam ob rem placuit ei, ut ad Ariovistum legatos mitteret, qui ab eo postularent, uti aliquem locum medium utriusque conloquio deligeret: velle sese de re publica et summis utriusque rebus cum eo agere. Ei legationi Ariovistus respondit: si quid ipsi a Caesare opus esset, sese ad eum venturum fuisse; si quid ille se velit, illum ad se venire oportere. Praeterea se neque sine exercitu in eas partes Galliae venire audere quas Caesar possideret, neque exercitum sine magno commeatu atque molimento in unum locum contrahere posse. Sibi autem mirum videri, quid in sua Gallia, quam bello vicisset, aut Caesari aut omnino populo Romano negotii esset.
		
}



\foreach \n in {0,...,40}{

	\lettrine[findent=5.0em]{H} is responsis ad Caesarem relatis, iterum ad eum Caesar legatos cum his mandatis mittit: quoniam tanto suo populique Romani beneficio adfectus, cum in consulatu suo rex atque amicus a senatu appellatus esset, hanc sibi populoque Romano gratiam referret, ut in conloquium venire invitatus gravaretur neque de communi re dicendum sibi et cognoscendum putaret, haec esse, quae ab eo postularet: primum ne quam multitudinem hominum amplius trans Rhenum in Galliam traduceret; deinde obsides, quos haberet ab Haeduis, redderet Sequanisque permitteret, ut, quos illi haberent, voluntate eius reddere illis liceret; neve Haeduos iniuria lacesseret neve his sociisque eorum bellum inferret. Si id ita fecisset, sibi populoque Romano perpetuam gratiam atque amicitiam cum eo futuram; si non impetraret, sese, quoniam M. Messala, M. Pisone consulibus senatus censuisset, uti quicumque Galliam provinciam obtineret, quod commodo rei publicae facere posset, Haeduos ceterosque amicos populi Romani defenderet, se Haeduorum iniurias non neglecturum.	
	
}

\renewcommand{\LettrineFontHook}{\color{brown}\GoudyInfamily{}}
%\LettrineTextFont{\itshape}
\setcounter{DefaultLines}{8}%

\begin{multicols}{2}
\foreach \n in {0,...,40}{

	\lettrine{A} d haec Ariovistus respondit: ius esse belli, ut, qui vicissent, iis, quos vicissent, quem ad modum vellent, imperarent. Item populum Romanum victis non ad alterius praescriptum, sed ad suum arbitrium imperare consuesse. Si ipse populo Romano non praescriberet quem ad modum suo iure uteretur, non oportere se a populo Romano in suo iure impediri. Haeduos sibi, quoniam belli fortunam temptassent et armis congressi ac superati essent, stipendiarios esse factos. Magnam Caesarem iniuriam facere, qui suo adventu vectigalia sibi deteriora faceret. Haeduis se obsides redditurum non esse neque his neque eorum sociis iniuria bellum inlaturum, si in eo manerent, quod convenisset, stipendiumque quotannis penderent; si id non fecissent, longe iis fraternum nomen populi Romani afuturum. Quod sibi Caesar denuntiaret se Haeduorum iniurias non neglecturum, neminem secum sine sua pernicie contendisse. Cum vellet, congrederetur: intellecturum quid invicti Germani, exercitatissimi in armis, qui inter annos XIIII tectum non subissent, virtute possent.	
	
}
\end{multicols}



\begin{multicols}{2}
\foreach \n in {0,...,40}{

	\lettrine{H} aec eodem tempore Caesari mandata referebantur et legati ab Haeduis et a Treveris veniebant: Haedui questum, quod Harudes, qui nuper in Galliam transportati essent, fines eorum popularentur: sese ne obsidibus quidem datis pacem Ariovisti redimere potuisse; Treveri autem, [I]pagos centum Sueborum ad ripas Rheni consedisse, qui Rhemum transire conarentur; his praeesse Nasuam et Cimberium fratres. Quibus rebus Caesar vehementer commotus maturandum sibi existimavit, ne, si nova manus Sueborum cum veteribus copiis Ariovisti sese coniunxisset, minus facile resisti posset. Itaque re frumentaria, quam celerrime potuit, comparata magnis itineribus ad Ariovistum contendit.
	
}
\end{multicols}



\begin{multicols}{2}
\foreach \n in {0,...,40}{

	\lettrine{C} um tridui viam processisset, nuntiatum est ei Ariovistum cum suis omnibus copiis ad occupandum Vesontionem, quod est oppidum maximum Sequanorum, contendere [triduique viam a suis finibus processisse]. Id ne accideret, magnopere sibi praecavendum Caesar existimabat. Namque omnium rerum, quae ad bellum usui erant, summa erat in eo oppido facultas, idque natura loci sic muniebatur, ut magnam ad ducendum bellum daret facultatem, propterea quod flumen Dubis ut circino circumductum paene totum oppidum cingit, reliquum spatium, quod est non amplius pedum MDC, qua flumen intermittit, mons continet magna altitudine, ita ut radices eius montis ex utraque parte ripae fluminis contingant, hunc murus circumdatus arcem efficit et cum oppido coniungit. Huc Caesar magnis nocturnis diurnisque itineribus contendit occupatoque oppido ibi praesidium conlocat.
	
}
\end{multicols}



\begin{multicols}{2}
\foreach \n in {0,...,40}{

	\lettrine{D} um paucos dies ad Vesontionem rei frumentariae commeatusque causa moratur, ex percontatione nostrorum vocibusque Gallorum ac mercatorum, qui ingenti magnitudine corporum Germanos, incredibili virtute atque exercitatione in armis esse praedicabant (saepe numero sese cum his congressos ne vultum quidem atque aciem oculorum dicebant ferre potuisse), tantus subito timor omnem exercitum occupavit, ut non mediocriter omnium mentes animosque perturbaret. Hic primum ortus est a tribunis militum, praefectis, reliquisque, qui ex urbe amicitiae causa Caesarem secuti non magnum in re militari usum habebant: quorum alius alia causa inlata, quam sibi ad proficiscendum necessariam esse diceret, petebat, ut eius voluntate discedere liceret; non nulli pudore adducti, ut timoris suspicionem vitarent, remanebant. Hi neque vultum fingere neque interdum lacrimas tenere poterant: abditi in tabernaculis aut suum fatum querebantur aut cum familiaribus suis commune periculum miserabantur. Vulgo totis castris testamenta obsignabantur. Horum vocibus ac timore paulatim etiam ii qui magnum in castris usum habebant, milites centurionesque quique equitatui praeerant, perturbabantur. Qui se ex his minus timidos existimari volebant, non se hostem vereri, sed angustias itineris et magnitudinem silvarum, quae intercederent inter ipsos atque Ariovistum, aut rem frumentariam, ut satis commode supportari posset, timere dicebant. Non nulli etiam Caesari nuntiabant, cum castra moveri ac signa ferri iussisset, non fore dicto audientes milites neque propter timorem signa laturos.
	
}
\end{multicols}

\renewcommand{\LettrineFontHook}{\color{brown}\GoudyInfamily{}}
\setcounter{DefaultLines}{4}%

\begin{multicols}{2}
\foreach \n in {0,...,40}{

	\lettrine{H} aec cum animadvertisset, convocato consilio omniumque ordinum ad id consilium adhibitis centurionibus, vehementer eos incusavit: primum, quod aut quam in partem aut quo consilio ducerentur sibi quaerendum aut cogitandum putarent. Ariovistum se consule cupidissime populi Romani amicitiam adpetisse; cur hunc tam temere quisquam ab officio discessurum iudicaret? Sibi quidem persuaderi cognitis suis postulatis atque aequitate condicionum perspecta eum neque suam neque populi Romani gratiam repudiaturum. Quod si furore atque amentia impulsum bellum intulisset, quid tandem vererentur? Aut cur de sua virtute aut de ipsius diligentia desperarent? Factum eius hostis periculum patrum nostrorum memoria Cimbris et Teutonis a C. Mario pulsis [cum non minorem laudem exercitus quam ipse imperator meritus videbatur]; factum etiam nuper in Italia servili tumultu, quos tamen aliquid usus ac disciplina, quam a nobis accepissent, sublevarint. Ex quo iudicari posse, quantum haberet in se boni constantia, propterea quod, quos aliquam diu inermes sine causa timuissent, hos postea armatos ac victores superassent. Denique hos esse eosdem Germanos, quibuscum saepe numero Helvetii congressi non solum in suis sed etiam in illorum finibus plerumque superarint, qui tamen pares esse nostro exercitui non potuerint. Si, quos adversum proelium et fuga Gallorum commoveret, hos, si quaererent, reperire posse diuturnitate belli defatigatis Gallis Ariovistum, cum multos menses castris se ac paludibus tenuisset neque sui potestatem fecisset, desperantes iam de pugna et dispersos subito adortum magis ratione et consilio quam virtute vicisse. Cui rationi contra homines barbaros atque imperitos locus fuisset, hac ne ipsum quidem sperare nostros exercitus capi posse. Qui suum timorem in rei frumentariae simulationem angustiasque itineris conferrent, facere arroganter, cum aut de officio imperatoris desperare aut praescribere viderentur. Haec sibi esse curae; frumentum Sequanos, Leucos, Lingones subministrare, iamque esse in agris frumenta matura; de itinere ipsos brevi tempore iudicaturos. Quod non fore dicto audientes neque signa laturi dicantur, nihil se ea re commoveri: scire enim, quibuscumque exercitus dicto audiens non fuerit, aut male re gesta fortunam defuisse aut aliquo facinore comperto avaritiam esse convictam. Suam innocentiam perpetua vita, felicitatem Helvetiorum bello esse perspectam. Itaque se, quod in longiorem diem conlaturus fuisset, repraesentaturum et proxima nocte de quarta vigilia castra moturum, ut quam primum intellegere posset, utrum apud eos pudor atque officium an timor plus valeret. Quod si praeterea nemo sequatur, tamen se cum sola decima legione iturum, de qua non dubitet, sibique eam praetoriam cohortem futuram. Huic legioni Caesar et indulserat praecipue et propter virtutem confidebat maxime.	
	
}
\end{multicols}

\begin{multicols}{2}
\foreach \n in {0,...,40}{

	\lettrine{H} ac oratione habita mirum in modum conversae sunt omnium mentes summaque alacritas et cupiditas belli gerendi innata est, princepsque X legio per tribunos militum ei gratias egit, quod de se optimum iudicium fecisset, seque esse ad bellum gerendum paratissimam confirmavit. Deinde reliquae legiones cum tribunis militum et primorum ordinum centurionibus egerunt uti Caesari satis facerent: se neque umquam dubitasse neque timuisse neque de summa belli suum iudicium sed imperatoris esse existimavisse. Eorum satisfactione accepta et itinere exquisito per Diviciacum, quod ex Gallis ei maximam fidem habebat, ut milium amplius quinquaginta circuitu locis apertis exercitum duceret, de quarta vigilia, ut dixerat, profectus est. Septimo die, cum iter non intermitteret, ab exploratoribus certior factus est Ariovisti copias a nostris milia passuum IIII et XX abesse.
	
}
\end{multicols}


\end{document}